\chapter{Tutorial}
\label{Tutorial}
\typeout{$Id$}

\section{Creating a Template Bundle}

To create information sheets for your game elements, you will need to
create templates\footnote{A pre-built template bundle for
\textit{Advanced Dungeond and Dragons}, dnd.rpgtmpl, is included, so if
you play using the \textit{Advanced Dungeond and Dragons} system, you
are all set to go.}.

\begin{figure}[hbpt] 
\begin{centering}
\includegraphics[width=5in]{OptionMenu.png} 
\caption{Selecting ``Create or edit a template file''} 
\label{fig:createoredittemplatemenu} 
\end{centering}
\end{figure} 
\begin{figure}[hbpt] 
\begin{centering}
\includegraphics{CreateOrEditTemplateDialog.png} 
\caption{The Create Or Edit Template Dialog}
\label{fig:createoredittemplate} 
\end{centering}
\end{figure} 
\begin{figure}[hbpt]
\begin{centering}
\includegraphics[width=5in]{EmptyTemplateEditor.png}
\caption{Empty Template Editor Window}
\label{fig:emptytemplate}
\end{centering}
\end{figure}
\begin{figure}[hbpt]
\begin{centering}
\includegraphics{AddNewTemplate.png}
\caption{Add New Template Dialog Box}
\label{fig:addnewtemplate}
\end{centering}
\end{figure}
\begin{figure}[hbpt]
\begin{centering}
\includegraphics{AddNewTemplatePlayer.png}
\caption{Add New Template Dialog Box, with ``Player'' filled in}
\label{fig:addnewtemplateplayer}
\end{centering}
\end{figure}
\begin{figure}[hbpt]
\begin{centering}
\includegraphics[width=5in]{PlayerTemplateEditor.png}
\caption{Template Editor, with empty ``Player'' template}
\label{fig:emptyplayertemplateeditor}
\end{centering}
\end{figure}
To create a template bundle, select ``Create or edit a template file''
from the Options menu, as shown in
Figure~\ref{fig:createoredittemplatemenu}.  This will display the dialog box
shown in Figure~\ref{fig:createoredittemplate}.  Select ``Create''.  An
empty template editor window, as shown in Figure~\ref{fig:emptytemplate}
will be opened up.  You can now start to create templates for your game
system.  We will create a simple Character class template.  First, click
on ``Add Template''.  This will open up the ``Add New Template'' dialog
box, as shown in Figure~\ref{fig:addnewtemplate}.  Type ``Player'' in
the name field, as shown in Figure~\ref{fig:addnewtemplateplayer}, and
click ``Add''.  This will create an entry under the ``Character'' folder
named ``Player''.  Double click on this entry now.  The template editor
will will now look like Figure~\ref{emptyplayertemplateeditor}.

\subsection{Adding heading text to a container}

\begin{figure}[hbpt]
\begin{centering}
\includegraphics{EmptyEditContainerText.png}
\caption{Empty ``Edit Container Text'' dialog box}
\label{fig:editcontainertext}
\end{centering}
\end{figure}
\begin{figure}[hbpt]
\begin{centering}
\includegraphics{EditContainerTextWithText.png}
\caption{``Edit Container Text'' dialog box with text added}
\label{fig:editcontainertextwithtext}
\end{centering}
\end{figure}
\begin{figure}[hbpt]
\begin{centering}
\includegraphics[width=5in]{PlayerTemplateEditorHeading.png}
\caption{Template Editor, with ``Player'' template, with a heading added}
\label{fig:playertemplateeditorwithheading}
\end{centering}
\end{figure}
At first, all that is in a sheet template is an empty toplevel
container, named for the class of sheet (Character) in this case.  The
first thing you will want to do is add a heading for this container.
Highlight the container name by clicking on it, then click the ``Edit
Continer Text'' button on the tool bar.  This will display the ``Edit
Container Text'' dialog box, as shown in
Figure~\ref{fig:editcontainertext}. Fill in the text field with ``This
is a player character''.  The dialog box will now look like
Figure~\ref{fig:editcontainertextwithtext}. Click ``Update''. The
template editor window will now look like
Figure~\ref{fig:playertemplateeditorwithheading}.

\subsection{Adding a field to a container}

\begin{figure}[hbpt]
\begin{centering}
\includegraphics{AddNewFieldDialog.png}
\caption{``Add New Field'' dialog box}
\label{fig:addnewfield}
\end{centering}
\end{figure}
\begin{figure}[hbpt]
\begin{centering}
\includegraphics{AddNewFieldDialogWithField.png}
\caption{``Add New Field'' dialog box with field values filled in}
\label{fig:addnewfieldfilledin}
\end{centering}
\end{figure}
\begin{figure}[hbpt]
\begin{centering}
\includegraphics[width=5in]{PlayerTemplateEditorWithField.png}
\caption{Template Editor, with ``Player'' template, with a field added}
\label{fig:playertemplateeditorwithfield}
\end{centering}
\end{figure}
To add a field to the toplevel container, make sure the container name
is highlighter (click on the name to be sure), and click on the ``Add
Field or Container''.  This will display the ``Add New Field'' dialog
box, shown in Figure~\ref{fig:addnewfield}.  Fill in the ``Name'' field
with ``Character Name'', select ``Word / Short Phrase'' from the
``Type'' menu, and set ``Updatable'' to ``no''.  The dialog box should
now look like Figure~\ref{fig:addnewfieldfilledin}.  Click the ``Add''
button on the dialog box.  The template editor window should now look
like Figure~\ref{fig:playertemplateeditorwithfield}.

\subsection{Adding a container to a container}

\begin{figure}[hbpt]
\begin{centering}
\includegraphics{AddNewFieldDialogWithContainer.png}
\caption{``Add New Field'' dialog box with container}
\label{fig:addnewcontainer}
\end{centering}
\end{figure}
\begin{figure}[hbpt]
\begin{centering}
\includegraphics[width=5in]{PlayerTemplateEditorWithContainer.png}
\caption{Template Editor, with ``Player'' template, with a container added}
\label{fig:playertemplateeditorwithcontainer}
\end{centering}
\end{figure}
To add a container to the toplevel container, make sure the container name
is highlighter (click on the name to be sure), and click on the ``Add
Field or Container''.  This will display the ``Add New Field'' dialog
box, shown in Figure~\ref{fig:addnewfield}.  Fill in the ``Name'' field
with ``Attributes'', select ``Container'' from the
``Type'' menu, and set ``Updatable'' to ``yes''.  The dialog box should
now look like Figure~\ref{fig:addnewcontainer}.  Click the ``Add''
button on the dialog box.  The template editor window should now look
like Figure~\ref{fig:playertemplateeditorwithcontainer}.

\section{Creating a character sheet}

\begin{figure}[hbpt]
\begin{centering}
\includegraphics{OpenConfigurationEditor.png}
\caption{Selecting ``Edit System Configuration'' from the ``Options'' menu}
\label{fig:opensysconfedit}
\end{centering}
\end{figure}
To create a character sheet, we first need to be sure that there is an
available template bundle.  Go to the \verb=Options= menu and select
\verb=Edit System Configuration=, as shown in
Figure~\ref{fig:opensysconfedit}. Click on the file folder button to the
right of the ``Template File'' field and navigate to the location of the
\verb=dnd.rpgtmpl= file included with the \thesystem. Click ``Open'' on
the file select dialog and then ``OK'' on the configuration editor
window.  You might want then go to the \verb=Options= menu and select
\verb=Save System Configuration= to write out this configuration.

\begin{figure}[hbpt]  
\begin{centering}
\includegraphics{SelectTemplateFile.png} 
\caption{The ``Select Template File'' dialog} 
\label{fig:selecttemplatefile} 
\end{centering}
\end{figure} 
\begin{figure}[hbpt] 
\begin{centering}
\includegraphics[width=5in]{EmptyPlayerCharacterSheet.png} 
\caption{An empty player character sheet} 
\label{fig:emptyplayercharacter}
\end{centering} 
\end{figure} 
Next, click on the \verb=Edit Character= button.  This will open the
``Open or Create Character'' dialog box, shown in
Figure~\ref{fig:opencreatechar}.  Click on the file folder button. 
This will open the ``Select Template File'' dialog, shown in
Figure~\ref{fig:selecttemplatefile}. Double click on ``Player.xml''. 
This will select the player template, rather than the default
non-player character (NPC) template.  Now click on the ``Create''
button on the ``Open or Create Character'' dialog box.  You should now
have an empty character sheet much like that shown in
Figure~\ref{fig:emptyplayercharacter}. You are now ready to create a
player character sheet!  The process is much like filling in a form.
Each piece of information is filled into a labeled space.  Numeric
values have small up and down arrows at the right end of the field and
you can either type in the numbers or use these arrows to increase or
decrease the value in the field.  Fields which take file names have a
folder button at the right end.  These buttons can be clicked on to
open a file browser to select the file\footnote{External files are
copied into the sheet bundle to allow for easy transport and sharing.}.
 Text areas will display a scroll bar once the amount of text grows to
be long enough to need it. The sheet is broken up into sections.  First
there is the character's full name and his or her nickname(s).  The
next section is the character's basic attributes: Strength,
Intellegence, Wisdom, Dexterity, Constitution, and Charisma.  Then
comes the character's demographics, which includes the characters race,
class, gender, age, and alignment.  Then the character's wealth and
health: gold pieces, hit points, experience points, and level.  Then
comes the extra detail, which includes a picture, a short bio, and a
full bio.  Finally there is information about the player, including the
player's name, address, phone number and E-Mail address.  Some of these
fields will be filled out with the help of your game master and some
fields will be filled in from dice rolls\footnote{The \thesystem\  does
not include a dice roll function, since it is expected that most
players would prefer to use their own dice or other source of random
numbers.}.

\section{Creating a map}



