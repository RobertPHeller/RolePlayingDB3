%* 
%* ------------------------------------------------------------------
%* Role PlayingDB V2.0 by Deepwoods Software
%* ------------------------------------------------------------------
%* Preface.tex - Preface
%* Created by Robert Heller on Wed Dec 30 11:23:20 1998
%* ------------------------------------------------------------------
%* Modification History: 
%* $Log: Preface.tex,v $
%* Revision 1.4  2000/10/09 22:29:32  heller
%* Fleshing out the text...
%*
%* Revision 1.3  2000/10/03 16:25:34  heller
%* Update Preface for commercial version.
%*
%* Revision 1.2  1999/07/14 22:17:34  heller
%* Eddy's Edits.
%*
%* Revision 1.1  1999/01/02 02:10:10  heller
%* Initial revision
%*
%* ------------------------------------------------------------------
%* Contents:
%* ------------------------------------------------------------------
%*  
%*     Role Playing DB -- A database package that creates and maintains
%* 		       a database of RPG characters, monsters, treasures,
%* 		       spells, and playing environments.
%* 
%*     Copyright (C) 1995,1998,1999  Robert Heller D/B/A Deepwoods Software
%* 			51 Locke Hill Road
%* 			Wendell, MA 01379-9728
%* 
%*     This program is free software; you can redistribute it and/or modify
%*     it under the terms of the GNU General Public License as published by
%*     the Free Software Foundation; either version 2 of the License, or
%*     (at your option) any later version.
%* 
%*     This program is distributed in the hope that it will be useful,
%*     but WITHOUT ANY WARRANTY; without even the implied warranty of
%*     MERCHANTABILITY or FITNESS FOR A PARTICULAR PURPOSE.  See the
%*     GNU General Public License for more details.
%* 
%*     You should have received a copy of the GNU General Public License
%*     along with this program; if not, write to the Free Software
%*     Foundation, Inc., 675 Mass Ave, Cambridge, MA 02139, USA.
%* 
%*  
%* 

\chapter*{Preface}
\markboth{PREFACE}{PREFACE}%
\typeout{$Id$}
\addcontentsline{toc}{chapter}{Preface}

RPGs\footnote{RPG: Role Playing Game, a game where the players take on
the roles of persons who might have lived (or may yet live) in a
different time and place.  See \cite{Gygax78,Gygax79}.} are a popular
pastime among many people these days.  Maybe they are a form of escape
from the rather mundane lives many people live, at least during the
workday.  An RPG allows the players to escape into a world where some
things are simpler, and some things more complex, in interesting
ways.

I have played AD\&D a few times and was dismayed at the amount of paperwork needed to keep track of everything.  Being a computer person, it
seemed to me that most of this paperwork could be replaced by a
computer and the information managed by a clever database system.  Given
that now there are high-powered laptop computers business people use
to keep track of and manage large corporations, it should be possible to
manage the odd imaginary universe on such a machine.  So I wrote
the \thesystem\ to manage all of the information that goes with an RPG.

The \thesystem\ maintains a database describing an RPG ``universe''.  
This ``universe'' contains a group of ``characters'', some player and
some non-player, a collection of ``monsters'', and one or more
``places'' (dungeons usually) where the ``monsters'' reside, generally
guarding some treasure.  The \thesystem\ helps game masters and players
keep track of the various things in the make-believe universe in
which the RPG takes place.

If you have {\em any} comments about this package, please let me know.
My electronic mail addresses are listed on the back side of the title
page.  My postal address is listed in Appendix~\ref{License}.  I would be
very interested in any comments users of the \thesystem\ package might
have.

\vspace{.25in}
\noindent
Robert Heller \\
Deepwoods Software \\
Wendell, MA, USA \\
January 1999

\section*{Addendum to the V2.1 manual}
\markboth{Preface}{Addendum to the V2.1 manual}%
\addcontentsline{toc}{section}{Addendum to the V2.1 manual}

After to talking to various people, I have made a number of upgrades to
the \thesystem, mostly colorful graphics.  I have also written in more
details into this user manual.

\section*{Addendum to the V3.0 manual}
\markboth{Preface}{Addendum to the V3.0 manual}%
\addcontentsline{toc}{section}{Addendum to the V3.0 manual}

This is a complete rewrite of the system.  Character, monster, spell, treasure,
trick/trap, and dressing ``sheets'' can be customized using a template
editor.  This allows the system to be used with any table-top RPG
system.  The data files are all ``bundled'' up as Zip archives
containing an XML file with the sheet information, plus any associated
media (graphics files or documents).  Template files are also Zip
archives containing an XML files that describe the various sheets.  Each
of these files is self-contained and can be carried from computer to
computer on the media of your choice (eg CD/DVD-Rs, thumb drives, flash
cards, etc.).  Map files are also Zip archives containing an XML files
along with any associated media (graphics files or documents).

\vspace{.25in}
\noindent
Robert Heller \\
Deepwoods Software \\
Wendell, MA, USA \\
October 2000
