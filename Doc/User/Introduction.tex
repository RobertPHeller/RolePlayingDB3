%* 
%* ------------------------------------------------------------------
%* Role PlayingDB V2.0 by Deepwoods Software
%* ------------------------------------------------------------------
%* IntroUserManual.tex - User Manual Introduction
%* Created by Robert Heller on Wed Dec 30 11:22:15 1998
%* ------------------------------------------------------------------
%* Modification History: 
%* $Log: IntroUserManual.tex,v $
%* Revision 1.4  2000/10/09 22:29:32  heller
%* Fleshing out the text...
%*
%* Revision 1.3  1999/07/14 23:23:46  heller
%* Small last minute update.
%*
%* Revision 1.2  1999/07/14 22:17:34  heller
%* Eddy's Edits.
%*
%* Revision 1.1  1999/01/02 02:11:05  heller
%* Initial revision
%*
%* ------------------------------------------------------------------
%* Contents:
%* ------------------------------------------------------------------
%*  
%*     Role Playing DB -- A database package that creates and maintains
%* 		       a database of RPG characters, monsters, treasures,
%* 		       spells, and playing environments.
%* 
%*     Copyright (C) 1995,1998  Robert Heller D/B/A Deepwoods Software
%* 			51 Locke Hill Road
%* 			Wendell, MA 01379-9728
%* 
%*     This program is free software; you can redistribute it and/or modify
%*     it under the terms of the GNU General Public License as published by
%*     the Free Software Foundation; either version 2 of the License, or
%*     (at your option) any later version.
%* 
%*     This program is distributed in the hope that it will be useful,
%*     but WITHOUT ANY WARRANTY; without even the implied warranty of
%*     MERCHANTABILITY or FITNESS FOR A PARTICULAR PURPOSE.  See the
%*     GNU General Public License for more details.
%* 
%*     You should have received a copy of the GNU General Public License
%*     along with this program; if not, write to the Free Software
%*     Foundation, Inc., 675 Mass Ave, Cambridge, MA 02139, USA.
%* 
%*  
%* 

\chapter{Introduction}
\label{Intro}
\typeout{$Id$}
\section{What Is the \thesystem?}

The \thesystem\ is a specialized database system with a GUI front end
designed to aid people who play RPGs.  Both the players and the masters
can find uses for this package, to manage the information that
describes the players' characters and the game environment and its
contents.

The system consists of a collection of Tcl/Tk (\cite{Ousterhout94})
script files that implement a GUI.  See \cite{HellerRPGTcl09} for a
detailed description of these script files.

\section{How this Manual Is Organized}

Most of this manual describes how to use the GUI.

The GUI has eight main toplevel GUI windows:

\begin{enumerate}

\item The {\em Main} window.  This is the main window and it is
described in detail in Section~\ref{Main}.  The main window is the main
start up screen and contains the means to navigate to other parts of the
program.

\item The {\em Sheet Template Editor} window.  This window is used to
edit the structure and contents of a ``sheet'' (Character, Monster,
Spell, Treasure, Trick / Trap, or Dressing).  This window is described
in detail in Section~\ref{Template}.

\item The {\em Character Editing} window.  This window is used to create
and edit Character Object data files and it is described in detail in
Section~\ref{Character}. 

\item The {\em Monster Editing} window.  This window is used to create
and edit Monster Object data files and it is described in detail in
Section~\ref{Monster}. 

\item The {\em Spell Editing} window.  This window is used to create
and edit Spell Object data files and it is described in detail in
Section~\ref{Spell}. 

\item The {\em Treasure Editing} window.  This window is used to create
and edit Treasure Object data files and it is described in detail in
Section~\ref{Treasure}. 

\item The {\em Trick / Trap Editing} window.  This window is used to create
and edit Trick or Trap Object data files and it is described in detail in
Section~\ref{TrickTrap}. 

\item The {\em Dressing Editing} window.  This window is used to create
and edit Dressing Object data files and it is described in detail in
Section~\ref{Dressing}. 

\item The {\em Map Editing} window.  This window is used to create
and edit Map Object data files and it is described in detail in
Section~\ref{Map}. 

\end{enumerate}


